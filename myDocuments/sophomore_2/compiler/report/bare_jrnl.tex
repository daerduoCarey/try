
%% bare_jrnl.tex
%% V1.3
%% 2007/01/11
%% by Michael Shell
%% see http://www.michaelshell.org/
%% for current contact information.
%%
%% This is a skeleton file demonstrating the use of IEEEtran.cls
%% (requires IEEEtran.cls version 1.7 or later) with an IEEE journal paper.
%%
%% Support sites:
%% http://www.michaelshell.org/tex/ieeetran/
%% http://www.ctan.org/tex-archive/macros/latex/contrib/IEEEtran/
%% and
%% http://www.ieee.org/

% *** Authors should verify (and, if needed, correct) their LaTeX system  ***
% *** with the testflow diagnostic prior to trusting their LaTeX platform ***
% *** with production work. IEEE's font choices can trigger bugs that do  ***
% *** not appear when using other class files.                            ***
% The testflow support page is at:
% http://www.michaelshell.org/tex/testflow/


%%*************************************************************************
%% Legal Notice:
%% This code is offered as-is without any warranty either expressed or
%% implied; without even the implied warranty of MERCHANTABILITY or
%% FITNESS FOR A PARTICULAR PURPOSE! 
%% User assumes all risk.
%% In no event shall IEEE or any contributor to this code be liable for
%% any damages or losses, including, but not limited to, incidental,
%% consequential, or any other damages, resulting from the use or misuse
%% of any information contained here.
%%
%% All comments are the opinions of their respective authors and are not
%% necessarily endorsed by the IEEE.
%%
%% This work is distributed under the LaTeX Project Public License (LPPL)
%% ( http://www.latex-project.org/ ) version 1.3, and may be freely used,
%% distributed and modified. A copy of the LPPL, version 1.3, is included
%% in the base LaTeX documentation of all distributions of LaTeX released
%% 2003/12/01 or later.
%% Retain all contribution notices and credits.
%% ** Modified files should be clearly indicated as such, including  **
%% ** renaming them and changing author support contact information. **
%%
%% File list of work: IEEEtran.cls, IEEEtran_HOWTO.pdf, bare_adv.tex,
%%                    bare_conf.tex, bare_jrnl.tex, bare_jrnl_compsoc.tex
%%*************************************************************************

% Note that the a4paper option is mainly intended so that authors in
% countries using A4 can easily print to A4 and see how their papers will
% look in print - the typesetting of the document will not typically be
% affected with changes in paper size (but the bottom and side margins will).
% Use the testflow package mentioned above to verify correct handling of
% both paper sizes by the user's LaTeX system.
%
% Also note that the "draftcls" or "draftclsnofoot", not "draft", option
% should be used if it is desired that the figures are to be displayed in
% draft mode.
%
\documentclass[journal]{IEEEtran}
%
\usepackage{listings}
% If IEEEtran.cls has not been installed into the LaTeX system files,
% manually specify the path to it like:
% \documentclass[journal]{../sty/IEEEtran}





% Some very useful LaTeX packages include:
% (uncomment the ones you want to load)


% *** MISC UTILITY PACKAGES ***
%
%\usepackage{ifpdf}
% Heiko Oberdiek's ifpdf.sty is very useful if you need conditional
% compilation based on whether the output is pdf or dvi.
% usage:
% \ifpdf
%   % pdf code
% \else
%   % dvi code
% \fi
% The latest version of ifpdf.sty can be obtained from:
% http://www.ctan.org/tex-archive/macros/latex/contrib/oberdiek/
% Also, note that IEEEtran.cls V1.7 and later provides a builtin
% \ifCLASSINFOpdf conditional that works the same way.
% When switching from latex to pdflatex and vice-versa, the compiler may
% have to be run twice to clear warning/error messages.






% *** CITATION PACKAGES ***
%
%\usepackage{cite}
% cite.sty was written by Donald Arseneau
% V1.6 and later of IEEEtran pre-defines the format of the cite.sty package
% \cite{} output to follow that of IEEE. Loading the cite package will
% result in citation numbers being automatically sorted and properly
% "compressed/ranged". e.g., [1], [9], [2], [7], [5], [6] without using
% cite.sty will become [1], [2], [5]--[7], [9] using cite.sty. cite.sty's
% \cite will automatically add leading space, if needed. Use cite.sty's
% noadjust option (cite.sty V3.8 and later) if you want to turn this off.
% cite.sty is already installed on most LaTeX systems. Be sure and use
% version 4.0 (2003-05-27) and later if using hyperref.sty. cite.sty does
% not currently provide for hyperlinked citations.
% The latest version can be obtained at:
% http://www.ctan.org/tex-archive/macros/latex/contrib/cite/
% The documentation is contained in the cite.sty file itself.






% *** GRAPHICS RELATED PACKAGES ***
%
\ifCLASSINFOpdf
  % \usepackage[pdftex]{graphicx}
  % declare the path(s) where your graphic files are
  % \graphicspath{{../pdf/}{../jpeg/}}
  % and their extensions so you won't have to specify these with
  % every instance of \includegraphics
  % \DeclareGraphicsExtensions{.pdf,.jpeg,.png}
\else
  % or other class option (dvipsone, dvipdf, if not using dvips). graphicx
  % will default to the driver specified in the system graphics.cfg if no
  % driver is specified.
  % \usepackage[dvips]{graphicx}
  % declare the path(s) where your graphic files are
  % \graphicspath{{../eps/}}
  % and their extensions so you won't have to specify these with
  % every instance of \includegraphics
  % \DeclareGraphicsExtensions{.eps}
\fi
% graphicx was written by David Carlisle and Sebastian Rahtz. It is
% required if you want graphics, photos, etc. graphicx.sty is already
% installed on most LaTeX systems. The latest version and documentation can
% be obtained at: 
% http://www.ctan.org/tex-archive/macros/latex/required/graphics/
% Another good source of documentation is "Using Imported Graphics in
% LaTeX2e" by Keith Reckdahl which can be found as epslatex.ps or
% epslatex.pdf at: http://www.ctan.org/tex-archive/info/
%
% latex, and pdflatex in dvi mode, support graphics in encapsulated
% postscript (.eps) format. pdflatex in pdf mode supports graphics
% in .pdf, .jpeg, .png and .mps (metapost) formats. Users should ensure
% that all non-photo figures use a vector format (.eps, .pdf, .mps) and
% not a bitmapped formats (.jpeg, .png). IEEE frowns on bitmapped formats
% which can result in "jaggedy"/blurry rendering of lines and letters as
% well as large increases in file sizes.
%
% You can find documentation about the pdfTeX application at:
% http://www.tug.org/applications/pdftex





% *** MATH PACKAGES ***
%
%\usepackage[cmex10]{amsmath}
% A popular package from the American Mathematical Society that provides
% many useful and powerful commands for dealing with mathematics. If using
% it, be sure to load this package with the cmex10 option to ensure that
% only type 1 fonts will utilized at all point sizes. Without this option,
% it is possible that some math symbols, particularly those within
% footnotes, will be rendered in bitmap form which will result in a
% document that can not be IEEE Xplore compliant!
%
% Also, note that the amsmath package sets \interdisplaylinepenalty to 10000
% thus preventing page breaks from occurring within multiline equations. Use:
%\interdisplaylinepenalty=2500
% after loading amsmath to restore such page breaks as IEEEtran.cls normally
% does. amsmath.sty is already installed on most LaTeX systems. The latest
% version and documentation can be obtained at:
% http://www.ctan.org/tex-archive/macros/latex/required/amslatex/math/





% *** SPECIALIZED LIST PACKAGES ***
%
%\usepackage{algorithmic}
% algorithmic.sty was written by Peter Williams and Rogerio Brito.
% This package provides an algorithmic environment fo describing algorithms.
% You can use the algorithmic environment in-text or within a figure
% environment to provide for a floating algorithm. Do NOT use the algorithm
% floating environment provided by algorithm.sty (by the same authors) or
% algorithm2e.sty (by Christophe Fiorio) as IEEE does not use dedicated
% algorithm float types and packages that provide these will not provide
% correct IEEE style captions. The latest version and documentation of
% algorithmic.sty can be obtained at:
% http://www.ctan.org/tex-archive/macros/latex/contrib/algorithms/
% There is also a support site at:
% http://algorithms.berlios.de/index.html
% Also of interest may be the (relatively newer and more customizable)
% algorithmicx.sty package by Szasz Janos:
% http://www.ctan.org/tex-archive/macros/latex/contrib/algorithmicx/




% *** ALIGNMENT PACKAGES ***
%
%\usepackage{array}
% Frank Mittelbach's and David Carlisle's array.sty patches and improves
% the standard LaTeX2e array and tabular environments to provide better
% appearance and additional user controls. As the default LaTeX2e table
% generation code is lacking to the point of almost being broken with
% respect to the quality of the end results, all users are strongly
% advised to use an enhanced (at the very least that provided by array.sty)
% set of table tools. array.sty is already installed on most systems. The
% latest version and documentation can be obtained at:
% http://www.ctan.org/tex-archive/macros/latex/required/tools/


%\usepackage{mdwmath}
%\usepackage{mdwtab}
% Also highly recommended is Mark Wooding's extremely powerful MDW tools,
% especially mdwmath.sty and mdwtab.sty which are used to format equations
% and tables, respectively. The MDWtools set is already installed on most
% LaTeX systems. The lastest version and documentation is available at:
% http://www.ctan.org/tex-archive/macros/latex/contrib/mdwtools/


% IEEEtran contains the IEEEeqnarray family of commands that can be used to
% generate multiline equations as well as matrices, tables, etc., of high
% quality.


%\usepackage{eqparbox}
% Also of notable interest is Scott Pakin's eqparbox package for creating
% (automatically sized) equal width boxes - aka "natural width parboxes".
% Available at:
% http://www.ctan.org/tex-archive/macros/latex/contrib/eqparbox/





% *** SUBFIGURE PACKAGES ***
%\usepackage[tight,footnotesize]{subfigure}
% subfigure.sty was written by Steven Douglas Cochran. This package makes it
% easy to put subfigures in your figures. e.g., "Figure 1a and 1b". For IEEE
% work, it is a good idea to load it with the tight package option to reduce
% the amount of white space around the subfigures. subfigure.sty is already
% installed on most LaTeX systems. The latest version and documentation can
% be obtained at:
% http://www.ctan.org/tex-archive/obsolete/macros/latex/contrib/subfigure/
% subfigure.sty has been superceeded by subfig.sty.



%\usepackage[caption=false]{caption}
%\usepackage[font=footnotesize]{subfig}
% subfig.sty, also written by Steven Douglas Cochran, is the modern
% replacement for subfigure.sty. However, subfig.sty requires and
% automatically loads Axel Sommerfeldt's caption.sty which will override
% IEEEtran.cls handling of captions and this will result in nonIEEE style
% figure/table captions. To prevent this problem, be sure and preload
% caption.sty with its "caption=false" package option. This is will preserve
% IEEEtran.cls handing of captions. Version 1.3 (2005/06/28) and later 
% (recommended due to many improvements over 1.2) of subfig.sty supports
% the caption=false option directly:
%\usepackage[caption=false,font=footnotesize]{subfig}
%
% The latest version and documentation can be obtained at:
% http://www.ctan.org/tex-archive/macros/latex/contrib/subfig/
% The latest version and documentation of caption.sty can be obtained at:
% http://www.ctan.org/tex-archive/macros/latex/contrib/caption/




% *** FLOAT PACKAGES ***
%
%\usepackage{fixltx2e}
% fixltx2e, the successor to the earlier fix2col.sty, was written by
% Frank Mittelbach and David Carlisle. This package corrects a few problems
% in the LaTeX2e kernel, the most notable of which is that in current
% LaTeX2e releases, the ordering of single and double column floats is not
% guaranteed to be preserved. Thus, an unpatched LaTeX2e can allow a
% single column figure to be placed prior to an earlier double column
% figure. The latest version and documentation can be found at:
% http://www.ctan.org/tex-archive/macros/latex/base/



%\usepackage{stfloats}
% stfloats.sty was written by Sigitas Tolusis. This package gives LaTeX2e
% the ability to do double column floats at the bottom of the page as well
% as the top. (e.g., "\begin{figure*}[!b]" is not normally possible in
% LaTeX2e). It also provides a command:
%\fnbelowfloat
% to enable the placement of footnotes below bottom floats (the standard
% LaTeX2e kernel puts them above bottom floats). This is an invasive package
% which rewrites many portions of the LaTeX2e float routines. It may not work
% with other packages that modify the LaTeX2e float routines. The latest
% version and documentation can be obtained at:
% http://www.ctan.org/tex-archive/macros/latex/contrib/sttools/
% Documentation is contained in the stfloats.sty comments as well as in the
% presfull.pdf file. Do not use the stfloats baselinefloat ability as IEEE
% does not allow \baselineskip to stretch. Authors submitting work to the
% IEEE should note that IEEE rarely uses double column equations and
% that authors should try to avoid such use. Do not be tempted to use the
% cuted.sty or midfloat.sty packages (also by Sigitas Tolusis) as IEEE does
% not format its papers in such ways.


%\ifCLASSOPTIONcaptionsoff
%  \usepackage[nomarkers]{endfloat}
% \let\MYoriglatexcaption\caption
% \renewcommand{\caption}[2][\relax]{\MYoriglatexcaption[#2]{#2}}
%\fi
% endfloat.sty was written by James Darrell McCauley and Jeff Goldberg.
% This package may be useful when used in conjunction with IEEEtran.cls'
% captionsoff option. Some IEEE journals/societies require that submissions
% have lists of figures/tables at the end of the paper and that
% figures/tables without any captions are placed on a page by themselves at
% the end of the document. If needed, the draftcls IEEEtran class option or
% \CLASSINPUTbaselinestretch interface can be used to increase the line
% spacing as well. Be sure and use the nomarkers option of endfloat to
% prevent endfloat from "marking" where the figures would have been placed
% in the text. The two hack lines of code above are a slight modification of
% that suggested by in the endfloat docs (section 8.3.1) to ensure that
% the full captions always appear in the list of figures/tables - even if
% the user used the short optional argument of \caption[]{}.
% IEEE papers do not typically make use of \caption[]'s optional argument,
% so this should not be an issue. A similar trick can be used to disable
% captions of packages such as subfig.sty that lack options to turn off
% the subcaptions:
% For subfig.sty:
% \let\MYorigsubfloat\subfloat
% \renewcommand{\subfloat}[2][\relax]{\MYorigsubfloat[]{#2}}
% For subfigure.sty:
% \let\MYorigsubfigure\subfigure
% \renewcommand{\subfigure}[2][\relax]{\MYorigsubfigure[]{#2}}
% However, the above trick will not work if both optional arguments of
% the \subfloat/subfig command are used. Furthermore, there needs to be a
% description of each subfigure *somewhere* and endfloat does not add
% subfigure captions to its list of figures. Thus, the best approach is to
% avoid the use of subfigure captions (many IEEE journals avoid them anyway)
% and instead reference/explain all the subfigures within the main caption.
% The latest version of endfloat.sty and its documentation can obtained at:
% http://www.ctan.org/tex-archive/macros/latex/contrib/endfloat/
%
% The IEEEtran \ifCLASSOPTIONcaptionsoff conditional can also be used
% later in the document, say, to conditionally put the References on a 
% page by themselves.





% *** PDF, URL AND HYPERLINK PACKAGES ***
%
%\usepackage{url}
% url.sty was written by Donald Arseneau. It provides better support for
% handling and breaking URLs. url.sty is already installed on most LaTeX
% systems. The latest version can be obtained at:
% http://www.ctan.org/tex-archive/macros/latex/contrib/misc/
% Read the url.sty source comments for usage information. Basically,
% \url{my_url_here}.





% *** Do not adjust lengths that control margins, column widths, etc. ***
% *** Do not use packages that alter fonts (such as pslatex).         ***
% There should be no need to do such things with IEEEtran.cls V1.6 and later.
% (Unless specifically asked to do so by the journal or conference you plan
% to submit to, of course. )


% correct bad hyphenation here
\hyphenation{op-tical net-works semi-conduc-tor}


\begin{document}
%
% paper title
% can use linebreaks \\ within to get better formatting as desired
\title{Java Implementation for C Compiler}
%
%
% author names and IEEE memberships
% note positions of commas and nonbreaking spaces ( ~ ) LaTeX will not break
% a structure at a ~ so this keeps an author's name from being broken across
% two lines.
% use \thanks{} to gain access to the first footnote area
% a separate \thanks must be used for each paragraph as LaTeX2e's \thanks
% was not built to handle multiple paragraphs
%

\author{Kaichun Mo,~\IEEEmembership{Shanghai Jiao Tong University}}

% note the % following the last \IEEEmembership and also \thanks - 
% these prevent an unwanted space from occurring between the last author name
% and the end of the author line. i.e., if you had this:
% 
% \author{....lastname \thanks{...} \thanks{...} }
%                     ^------------^------------^----Do not want these spaces!
%
% a space would be appended to the last name and could cause every name on that
% line to be shifted left slightly. This is one of those "LaTeX things". For
% instance, "\textbf{A} \textbf{B}" will typeset as "A B" not "AB". To get
% "AB" then you have to do: "\textbf{A}\textbf{B}"
% \thanks is no different in this regard, so shield the last } of each \thanks
% that ends a line with a % and do not let a space in before the next \thanks.
% Spaces after \IEEEmembership other than the last one are OK (and needed) as
% you are supposed to have spaces between the names. For what it is worth,
% this is a minor point as most people would not even notice if the said evil
% space somehow managed to creep in.



% The paper headers
\markboth{Compiler Report}%
{Java Implementation for C Compiler}
% The only time the second header will appear is for the odd numbered pages
% after the title page when using the twoside option.
% 
% *** Note that you probably will NOT want to include the author's ***
% *** name in the headers of peer review papers.                   ***
% You can use \ifCLASSOPTIONpeerreview for conditional compilation here if
% you desire.




% If you want to put a publisher's ID mark on the page you can do it like
% this:
%\IEEEpubid{0000--0000/00\$00.00~\copyright~2007 IEEE}
% Remember, if you use this you must call \IEEEpubidadjcol in the second
% column for its text to clear the IEEEpubid mark.



% use for special paper notices
%\IEEEspecialpapernotice{(Invited Paper)}




% make the title area
\maketitle


\begin{abstract}
%\boldmath
This is the report for my compiler project in spring of 2014.
\end{abstract}
% IEEEtran.cls defaults to using nonbold math in the Abstract.
% This preserves the distinction between vectors and scalars. However,
% if the journal you are submitting to favors bold math in the abstract,
% then you can use LaTeX's standard command \boldmath at the very start
% of the abstract to achieve this. Many IEEE journals frown on math
% in the abstract anyway.

% Note that keywords are not normally used for peerreview papers.
\begin{IEEEkeywords}
Compiler Report, Syntactic Analysis, Semantic Analysis, Parser, Intermediate Code, Register Allocation, Optimizations, Function Inline, Dead Code Elimination
\end{IEEEkeywords}



% For peer review papers, you can put extra information on the cover
% page as needed:
% \ifCLASSOPTIONpeerreview
% \begin{center} \bfseries EDICS Category: 3-BBND \end{center}
% \fi
%
% For peerreview papers, this IEEEtran command inserts a page break and
% creates the second title. It will be ignored for other modes.
\IEEEpeerreviewmaketitle



\section{Introduction}
% The very first letter is a 2 line initial drop letter followed
% by the rest of the first word in caps.
% 
% form to use if the first word consists of a single letter:
% \IEEEPARstart{A}{demo} file is ....
% 
% form to use if you need the single drop letter followed by
% normal text (unknown if ever used by IEEE):
% \IEEEPARstart{A}{}demo file is ....
% 
% Some journals put the first two words in caps:
% \IEEEPARstart{T}{his demo} file is ....
% 
% Here we have the typical use of a "T" for an initial drop letter
% and "HIS" in caps to complete the first word.
\IEEEPARstart{T}{his} is the report for my compiler project in spring of 2014.  In this report, I mainly discuss about the implementation details when dealing with syntactic analysis, semantic analysis, translation to intermediate code and many reasonable optimizations.


\section{Syntactic and Semantic Analysis}

\subsection{Lexical Analysis}

I use \textit{Jflex} tool to generate my lexer. Special attention should be devoted to how to deal with string constant and how to ignore comments and includes.

\subsubsection{Ignorance of Includes}

I just use DFA to match includes and just ignore them.
\begin{scriptsize}
\begin{lstlisting}
IncludeComment = ("#" {White}* "include" {White}* "<"
 [a-zA-Z.]* ">")|("#" {White}*"include" {White}* "\"" 
 [a-zA-Z.]* "\"")
\end{lstlisting}
\end{scriptsize}
where 
\begin{scriptsize}
\begin{lstlisting}
                       White = [ \t\f]
\end{lstlisting}
\end{scriptsize}

\subsubsection{Ignorance of Comments and Recognize String Constant}
Just set up new yystate to record all the content in comments or strings. Nothing special.

\subsection{Parser}

I use \textit{Java\_Cup} tool to generate my parser. Special attention should be devoted to how to deal with $*$ or $+$.

For the grammar below.
\begin{center}
A$\rightarrow$B*
\end{center}
I use two different translation methods to ease the later coding. 

\subsubsection{Collect as a List} Look at the following code.

\begin{scriptsize}
\begin{lstlisting}
A ::= B_list:l B:e {: l.list.add(e); RESULT = l; :}
    | B:e	   {: RESULT = new B_list(e);	 :}
    ;
\end{lstlisting}
\end{scriptsize}
Using this way, we can collect all the $B$'s in a LinkedList, which can make it easier to sequentially traverse all the $B$'s.

\subsubsection{Collect as a tree} Another way is to generate the elements as a tree.
\begin{scriptsize}
\begin{lstlisting}
A ::= B_list:l B:e {: RESULT = new A(l, e); :}
    | B:e	   {: RESULT = new A(e);    :}
    ;
\end{lstlisting}
\end{scriptsize}
This method are widely used when dealing with binary expression. Since the operands of a binary expression has potential priority during execution, presenting binary expression as a tree is a better option.

\subsection{Semantic Analysis}

The job of this phase is to give each variable a type and consequently check type matching when necessary aiming to guarantee the validity of the program. The job is easy to do besides the following challenges.

\subsubsection{Type for Array} 
Consider the following declaration for array \textit{a}.
\begin{center}
\texttt{int a[10][20];}
\end{center}
We should set 
\begin{center}
\textit{a}: Array(Array(int, 20), 10) ($\surd$)
\end{center}
instead of
\begin{center}
\textit{a}: Array(Array(int, 10), 20) ($\times$)
\end{center}

\subsubsection{Operation between Pointer and Integer}

I set up a list of rules to judge whether a pointer type variable can do specific operation with an integer. For example,
\begin{itemize}
\item pointer-pointer ($\surd$)
\item pointer-integer ($\surd$)
\item pointer+pointer ($\times$)
\item integer-pointer ($\times$)
\item etc.
\end{itemize}

\subsubsection{Nested Definition for Struct}

Since the struct does not mean the beginning of a new environment, it's not valid that
\begin{lstlisting}[language=c]
	struct A
	{
		struct A
		{
			int x;
		} y;
	};
\end{lstlisting}
After noticing this problem, it can be solved easily.

\subsubsection{Struct Used before Declared}

Since the following used before fully declared is valid, we need to deal with this problem specially.
\begin{lstlisting}[language=c]
	struct A
	{
		int x;
		struct A *next;
	};
\end{lstlisting}
I solve this problem simply by rechecking. During the definition for struct \textit{A}, I firstly give the variable \textit{next} a type named \texttt{Name}. Immediately after the definition for \textit{A} is finished, I just simply recheck all the fields in struct \textit{A} and modify all \texttt{Name} type variable to its regular type with \textit{A} involved.

\subsubsection{Constant Computation}

Operations between constants can be executed during compile time. For example, 
\begin{center}
\texttt{int a[8+8-1];}
\end{center}
where $8+8-1=15$ is necessary when determining the size of \textit{a}.

\section{Intermediate Representation}

\IEEEPARstart{I}{} generate my own intermediate representation aiming at to ease the translation from IR to final MIPS code as well as retaining a relatively large possibility for back-end optimizations. 

I create a Java class named \texttt{IRStmt} as the super-class for all IR statements. While, a class named \texttt{Temp} is created to store all temporary variables in the IR code and a class named \texttt{Label} is created to remember the information for labels, including function labels and normal branch labels.

Similar to the MIPS code style, my IR statements are classified to the following 27 quadruples.

\subsection{Conditional Branch Instructions}

\subsubsection{Beq} 
Contain two \texttt{Temp}s \textit{reg1}, \textit{reg2} and one \texttt{Label} \textit{label} which means \textit{beq reg1, reg2, label}.

\subsubsection{Bne} 
Contain two \texttt{Temp}s \textit{reg1}, \textit{reg2} and one \texttt{Label} \textit{label} which means \textit{bne reg1, reg2, label}.

\subsubsection{Beqz} 
Contain one \texttt{Temp} \textit{reg} and one \texttt{Label} \textit{label} which means \textit{beqz reg label}.

\subsubsection{Bnez} 
Contain one \texttt{Temp} \textit{reg} and one \texttt{Label} \textit{label} which means \textit{bnez reg label}.

\subsection{Expression Computation Instructions}

\subsubsection{BinaryOp}
Contain one binary operator \textit{op} and three \texttt{Temp}s \textit{res}, \textit{left}, \textit{right} which means \textit{BOp res, left, right}, where \textit{BOp} are all possible binary operation instruction, just as \texttt{add}, \texttt{sub}, \texttt{sll}, etc.

\subsubsection{BinaryOpI}
Contain one binary operator \textit{op}, two \texttt{Temp}s \textit{res}, \textit{left} and one immediate number \texttt{right} which means \textit{BOpI res, left, right}, where \textit{BOpI} can be \texttt{addi}, \texttt{andi}, \texttt{xori}, etc.

\subsubsection{UnaryOp}
Contain one unary operator \textit{op} and two \texttt{Temp}s \textit{res}, \textit{reg} which means \textit{UOp res, reg}, where \textit{UOp} are all possible unary operation instruction, just as \texttt{neg}, \texttt{not}, etc.

\subsubsection{Compare}
Contain one comparison operator \textit{op} and three \texttt{Temp}s \textit{res}, \textit{left}, \textit{right} which means \textit{COp res, left, right}, where \textit{COp} can be \texttt{slt}, \texttt{sgt}, \texttt{sltu}, etc.

\subsubsection{CompareI}
Contain one comparison operator \textit{op}, two \texttt{Temp}s \textit{res}, \textit{left} and one immediate number \textit{right} which means \textit{COpI res, left, right}, where \textit{COpI} can only be \texttt{slti} and \texttt{sltiu}.

\subsection{Unconditional Jump Instructions}

\subsubsection{\ Jump}
Contain one \texttt{Label} \textit{label} which is the target of this jump instruction. This IR statement can be directly translate to MIPS code \textit{j label}.

\subsubsection{\ JumpReg}
Contain one \texttt{Temp} \textit{add} which stores the target of this jump to register instruction. This IR statement can be directly translate to MIPS code \textit{jr reg}.

\subsubsection{\ Jal}
Contain one \texttt{Label} \textit{label} which is the target of this jump and link instruction. This IR statement can be directly translate to MIPS code \textit{jal label}.

\subsubsection{\ JalR}
Contain one \texttt{Temp} \textit{reg} which stores the target of this jump and link to register instruction. This IR statement can be directly translate to MIPS code \textit{jalr reg}.

\subsection{Load/Store/Move Instructions}

\subsubsection{\ LoadA}
Contain one \texttt{Temp} \textit{res} and one \texttt{Label} \textit{label} which stores the address of the data. This IR statement corresponds to MIPS code \textit{la res, label}.

\subsubsection{\ LoadI}
Contain one \texttt{Temp} \textit{res} and one immediate number \textit{number} which corresponds to MIPS code \textit{li res, number}.

\subsubsection{\ LoadW}
Contain two \texttt{Temp}s \textit{res}, \textit{add} and one immediate number \textit{offset} which corresponds to MIPS code \textit{lw res, offset(add)}.

\subsubsection{\ LoadB}
Contain two \texttt{Temp}s \textit{res}, \textit{add} and one immediate number \textit{offset} which corresponds to MIPS code \textit{lb res, offset(add)}.

\subsubsection{\ Move}
Contain two \texttt{Temp}s \textit{res}, \textit{ori} which corresponds to MIPS code \textit{move res, ori}.

\subsubsection{\ StoreW}
Contain two \texttt{Temp}s \textit{res}, \textit{add} and one immediate number \textit{offset} which corresponds to MIPS code \textit{sw res, offset(add)}.

\subsubsection{\ StoreB}
Contain two \texttt{Temp}s \textit{res}, \textit{add} and one immediate number \textit{offset} which corresponds to MIPS code \textit{sb res, offset(add)}.

\subsection{Function Call Related Instructions}

\subsubsection{\ FuncCall}
Contain one \texttt{Temp} \textit{res} to store the possible return value of this function. This class also contain \texttt{FuncEntry} \textit{entry} which contains all the information about this function.

\subsubsection{\ Param}
This class is designed to prepare the function arguments before function call. 
For example, if \texttt{a=f(b,c)}
shows in the C code, I will generate the following IR codes.
\begin{center}
\texttt{Param $reg_b$}  \\
\texttt{Param $reg_c$} \\
\texttt{$reg_a$ = f()} 
\end{center}
This method can help me handle function parameters better when translating to final MIPS code.

\subsubsection{\ getParam}
This class is designed to receive the function parameters in. Using the same example above, in code segment for $f$, I will generate the following IR codes first.
\begin{center}
\texttt{f: \ \  \ \ \ \  \ \  \ \  \ \  \ \ } \\
\texttt{getParam t1}  \\
\texttt{getParam t2} \\
\end{center}
to get the parameter $b$ and $c$ into temporary \texttt{Temp} \textit{t1} and \textit{t2} respectively.

\subsubsection{\ Return}
This class contains a single \texttt{Temp} \textit{res} which means \textit{move \$v0, res} to return the function return value in MIPS code.

\subsection{Other Useful Instructions}

\subsubsection{\ LabelStmt}
This class contains one \texttt{Label} \textit{label} just to print \textit{label:} in IR code.

\subsubsection{\ Syscall}
This class just corresponds to MIPS code $syscall$. This class can be used when inlining the function \textit{printf}.

\subsubsection{\ Nop}
This class just stands for the ending of a function. When I meet a \texttt{Nop} when doing MIPS translation, I can begin dealing with the final procedure of this function, just as load the original \textit{fp} and \textit{ra} back, adjust the \textit{sp}, \textit{fp}, etc.

\section{Final MIPS Translation}

\IEEEPARstart{M}{ore} methods should be mentioned when translating from IR code to the final MIPS code. 

\subsection{Using MIPS Directives}

For better using the features provided by MIPS, I put all the global variable in data segment. This can help me better handle the global variables from one side, but this idea bring me some troubles when doing register allocation for global variables from another side.

I create a class named \texttt{Directives} as a super-class for all directives. I use 5 classes to handle this problem.

\subsection*{Directives Classes}

\subsubsection{DotAsciiz}
For global string constant, just like 
\begin{center}
\texttt{char *s="Hello, world!"}
\end{center}
 I use this class to store the string as follows.
\begin{center}
\texttt{L1: \ \  \ \ \ \ \ \  \  \ \ \ \ \ \  \   \  \ \ \ \ \ \  \ \  \ } \\
\texttt{.asciiz "Hello, world!"} \\
\end{center}

\subsubsection{DotByte}
\texttt{.byte} can be used to store a global character.

\subsubsection{DotWord}
\texttt{.word} can be used to store some consequent numbers.

\subsubsection{DotSpace}
\texttt{.space} can be used to apply for some blank continuous space.

\subsubsection{DotLabel}
This class is similar to \texttt{LabelStmt} in above quadruples. The only difference is that this class extends \texttt{Directives} rather than \texttt{IRStmt}.

Using the directives can help me better handle the global variables. When I need its address, I can use \texttt{la reg, label} to receive its address and use \texttt{lw reg, label} to get its value in final MIPS code.

\subsection{Dealing with Struct Return and Struct Parameter Passing}

\subsubsection{Struct Return}
If a function return a struct or union, I take the following method to deal with this problem. I pass one more parameter to this function, which is the start address pointing at a blank space of this struct size. When the function finishes computing the result, it chooses to directly write the answer in this space, instead of return the struct instance back.

\subsubsection{Struct Parameter Passing}
I can just put the whole struct content at the parameter area as what we do to normal parameters. And, what we need to do more is just to tell the function the information about each parameter's size and its start address in the memory.

\subsection{Dealing with Printf and Malloc}

At first, I write two MIPS function \texttt{printf} and \texttt{malloc} to implement these two needed functions. But, to optimize the performance, it's compelling to inline these two functions. 

\subsubsection{Malloc}
When we need to allocate some space for a variable, we can use the following codes to do that.
\begin{center}
\texttt{
	li \$v0, 9  \ \ \  \ \ \ \ \\
	li \$a0, size \ \ \ \\
 	syscall  \ \ \  \ \ \ \ \ \\
	move reg, \$v0 \ \ \\
}
\end{center}
where the \texttt{Temp} \textit{reg} store the address of this space.

\subsubsection{Printf}
When we encounter with the following code.
\begin{center}
\texttt{printf("1:\%d, 2: \%c end.",a,b);}
\end{center}
We can automatically split this into three instructions during compile time, if the print format is a string constant.
\begin{center}
\texttt{
	print\_str("1:") \ \ \ \ \ \\
	print\_num(a) \ \ \ \ \ \ \ \\
	print\_str(", 2: ") \\
	print\_char(b) \ \ \ \ \ \ \\
	print\_str(" end.") \\
}
\end{center}
where each little function can be implemented easily using \texttt{syscall}.

Thus, using this idea, it's quite easy to inline \texttt{printf} as much as possible. But, if \texttt{"\%0nd"} occurs or the print format is not constant, we cannot inline \texttt{printf}, I just simply call my MIPS \texttt{printf} procedure when suffering those situations.

\section{Optimizations}

\IEEEPARstart{T}{o} strengthen my compiler, I implement some classical optimizations as well as some little but powerful ones. Here are they.

\subsection{Register Allocation}
This is the most powerful optimization before generating final MIPS code. I use \textit{Linear Scan}[4] Method to do this. It is a general and compulsory optimization when implementing compilers. There is no need to describe the procedure. The method can be easily access from any compiler book.

I use 19 registers when doing linear scan, and I only allocate for local variables. Since I do not use SSA form IR, to guarantee the correctness, I do the allocation procedure in every basic block and connect the alive sections for each temporary variable together as the alive section of this variable. For example, if variable \textit{t1} has two separate sections $[3,10]$ and $[15,30]$, I take $[3,30]$ as the alive section for \textit{t1}. This looks like a waste of resources, but it works well if the program do not need too many variables.

\subsection{Function Inline}

I implement function inline for the funtions satisfies all the following properties.
\begin{itemize}
\item Its size is of small scale.
\item It does not have any declared variables.
\item It does not call any other functions.
\item Its whole parameters and return value are not of struct or union type.
\end{itemize}

I follow the procedure below to do the function inline.

\subsubsection*{\textbf{Input}}
IR codes

\subsubsection*{\textbf{Output}}
optimized IR codes

\subsubsection*{\textbf{Procedure}}
\begin{itemize}
\item Compute out all the functions that can be inlined and include them in set $S$.
\item Locate all function calls whose corresponding function is $s\in S$.
\item Rename all temporary variables in $s$ in order to eliminate the name conflicts after inlining.
\item Replace all formal parameter variables in $s$ with real variables in $f$, where $f$ is the function that calls $s$.
\end{itemize}

For example, the following code can do function inline.
\begin{center}
\texttt{
f:  \ \  \ \ \ \  \ \ \ \  \ \ \  \  \ \ \ \ \ \ \\
getParam \$t1  \  \ \ \  \   \\
addi \$t2, \$t1, 1 \ \\
Return \$t2  \ \ \ \  \ \  \ \\
g:  \ \  \ \ \ \  \ \ \ \  \ \ \  \  \ \ \ \ \ \\
Param \$t3 \  \ \ \  \ \  \ \ \\
\$t4=f() \  \ \ \ \ \ \ \  \ \  \\\
}
\end{center}

The optimized IR code is as follows.
\begin{center}
\texttt{
g:  \ \  \ \ \ \  \ \ \ \  \ \ \  \  \ \ \ \ \ \  \\
addi \$t5, \$t3, 1\\
\$t4=\$t5 \  \ \ \ \ \ \ \  \ \  \\\
}
\end{center}
 
Using the function inline method, my MIPS code can be more efficient.

\subsection{Global Variable Optimization}

Since my dealing with global variables is not so good and all the global variables need load before use, I come up with some ideas to optimize this problem.

\subsubsection{Optimization in Basic Block}

In one basic block, one global variable can be used more than one time. Since its address is fixed, I can just retain the first occurrence of \texttt{LoadA} instructions and delete all others. But after this simple modification, I need to modify all the registers storing its address to the one register survived. For example, I can optimize the following codes to a better way.
\begin{center}
\texttt{
la \$t0, L1 \ \ \ \\
lw \$t1, \$t0 \\
la \$t2, L1 \ \ \\
lw \$t3, \$t2 \\
}
\end{center}
After the optimization, I get a better code below.
\begin{center}
\texttt{
la \$t0, L1 \ \ \ \\
lw \$t1, \$t0 \\
lw \$t3, \$t0 \\
}
\end{center}

After using this optimization, I find the performance get better if there are relatively large number of global variables involved.

\subsubsection{Optimization in Functions}

If a function has many global variables involved and it does not have function calls to another functions, I can just load all the global variables' addresses at the first place when entering this function. This optimization works considerably for a small scale of programs in the test set.

\subsection{Dead Code Elimination}

Dead code can be detected during any phase of coding my compiler.

\subsubsection{Useless Variables}

When doing register allocation, we can find some useless temporary variables. We can just simply remove them from the IR code without affecting the result. The useless variable should satisfy the following property.
\begin{itemize}
\item The variable is defined, but it is never used.
\end{itemize}

For example, when we encounter with 
\begin{center}
\texttt{i++;}
\end{center}
we just simply translate it into 
\begin{center}
\texttt{
move \$t0, reg \ \  \ \ \ \\
addi reg, reg, 1 \\
}
\end{center}
where \textit{reg} stores the original value of \textit{i} and \textit{\$t0} is created for possible subsequent use the original value of \textit{i}. But, since \textit{\$t0} is defined but never used, my compiler just ignore the \texttt{move} statement. This makes sense, because the purpose of this code is just to increase the value of \textit{i}.

\subsubsection{Redundant Statement}

If the following code occurs,
\begin{center}
\texttt{
if(0==1)
{
	[stmt1]
} \\
else
{
	[stmt2]
}
} \ \ 
\end{center}
there is no way [stmt1] can be executed. Thus, just remove the garbage code and keep the useful code can save more execution time. 

After the optimization, we get
\begin{center}
\texttt{
[stmt2]
}
\end{center}
which is more efficient and concise.

\subsection{Other trivial Optimization}

\subsubsection{Pseudo-instructions Elimination}

Sometimes, using pseudo-instructions can waste more time. For example, 
\begin{center}
\texttt{seq \$t0, \$t1, \$t2}
\end{center}
are translated to four real instructions in SPIM simulator. But after thinking a little bit, I find that using two instructions can implement \texttt{seq}, which contributes more efficiency.
\begin{center}
\texttt{
xor \$t0, \$t1, \$t2 \ \ \\
sltiu \$t0, \$t0, 1 \\
}
\end{center}

The same optimizations can be done to \texttt{sne}, \texttt{sge} and \texttt{sle}.

\subsubsection{Short-circuit Optimization}

When we encounter with 
\begin{center}
\texttt{
a \&\& b \&\& c
}
\end{center}
it's very natural to do the translation as follows.
\begin{center}
\texttt{
t1 = 1 \ \ \ \ \ \ \ \ \ \ \ \\
	beqz a goto L1 \ \ \\
	t3 = 1 \ \ \ \ \ \ \ \ \ \ \\
	beqz b goto L3 \ \ \\
 	beqz c goto L3 \ \ \\
	j L4 \ \ \ \ \ \ \ \ \ \ \ \ \\
L3: \ \ \ \ \ \ \ \ \ \ \ \ \ \ \ \ \ \ \ \ \\
	t3 = 0 \ \ \ \ \ \ \ \ \ \ \\
L4: \ \ \ \ \ \ \ \ \ \ \ \ \ \ \ \ \ \ \ \ \\
beqz t3 goto L1 \\
	j L2 \ \ \ \ \ \ \ \ \ \ \ \ \\
L1: \ \ \ \ \ \ \ \ \ \ \ \ \ \ \ \ \ \ \ \ \\
	t1 = 0 \ \ \ \ \ \ \ \ \ \ \\
L2: \ \ \ \ \ \ \ \ \ \ \ \ \ \ \ \ \ \ \ \ \\
}
\end{center}
But the efficiency is terrible, compared to the following optimized code.
\begin{center}
\texttt{
t2 = 1 \ \ \ \ \ \ \ \ \ \  \\
beqz a goto L1 \\
	beqz b goto L1 \\
 	beqz c goto L1 \\
	j L2\ \ \ \ \ \ \ \ \ \ \ \ \\ 
L1: \ \ \ \ \ \ \ \ \ \ \ \ \ \ \ \ \ \ \\
	t2 = \$0 \ \ \ \ \ \ \ \ \\
L2:\ \ \ \ \ \ \ \ \ \ \ \ \ \ \ \ \ \ \ \\
}
\end{center}	
which uses Short-circuit Optimization.

%\subsection{Subsection Heading Here}
%Subsection text here.

% needed in second column of first page if using \IEEEpubid
%\IEEEpubidadjcol

%\subsubsection{Subsubsection Heading Here}
%Subsubsection text here.


% An example of a floating figure using the graphicx package.
% Note that \label must occur AFTER (or within) \caption.
% For figures, \caption should occur after the \includegraphics.
% Note that IEEEtran v1.7 and later has special internal code that
% is designed to preserve the operation of \label within \caption
% even when the captionsoff option is in effect. However, because
% of issues like this, it may be the safest practice to put all your
% \label just after \caption rather than within \caption{}.
%
% Reminder: the "draftcls" or "draftclsnofoot", not "draft", class
% option should be used if it is desired that the figures are to be
% displayed while in draft mode.
%
%\begin{figure}[!t]
%\centering
%\includegraphics[width=2.5in]{myfigure}
% where an .eps filename suffix will be assumed under latex, 
% and a .pdf suffix will be assumed for pdflatex; or what has been declared
% via \DeclareGraphicsExtensions.
%\caption{Simulation Results}
%\label{fig_sim}
%\end{figure}

% Note that IEEE typically puts floats only at the top, even when this
% results in a large percentage of a column being occupied by floats.


% An example of a double column floating figure using two subfigures.
% (The subfig.sty package must be loaded for this to work.)
% The subfigure \label commands are set within each subfloat command, the
% \label for the overall figure must come after \caption.
% \hfil must be used as a separator to get equal spacing.
% The subfigure.sty package works much the same way, except \subfigure is
% used instead of \subfloat.
%
%\begin{figure*}[!t]
%\centerline{\subfloat[Case I]\includegraphics[width=2.5in]{subfigcase1}%
%\label{fig_first_case}}
%\hfil
%\subfloat[Case II]{\includegraphics[width=2.5in]{subfigcase2}%
%\label{fig_second_case}}}
%\caption{Simulation results}
%\label{fig_sim}
%\end{figure*}
%
% Note that often IEEE papers with subfigures do not employ subfigure
% captions (using the optional argument to \subfloat), but instead will
% reference/describe all of them (a), (b), etc., within the main caption.


% An example of a floating table. Note that, for IEEE style tables, the 
% \caption command should come BEFORE the table. Table text will default to
% \footnotesize as IEEE normally uses this smaller font for tables.
% The \label must come after \caption as always.
%
%\begin{table}[!t]
%% increase table row spacing, adjust to taste
%\renewcommand{\arraystretch}{1.3}
% if using array.sty, it might be a good idea to tweak the value of
% \extrarowheight as needed to properly center the text within the cells
%\caption{An Example of a Table}
%\label{table_example}
%\centering
%% Some packages, such as MDW tools, offer better commands for making tables
%% than the plain LaTeX2e tabular which is used here.
%\begin{tabular}{|c||c|}
%\hline
%One & Two\\
%\hline
%Three & Four\\
%\hline
%\end{tabular}
%\end{table}


% Note that IEEE does not put floats in the very first column - or typically
% anywhere on the first page for that matter. Also, in-text middle ("here")
% positioning is not used. Most IEEE journals use top floats exclusively.
% Note that, LaTeX2e, unlike IEEE journals, places footnotes above bottom
% floats. This can be corrected via the \fnbelowfloat command of the
% stfloats package



% if have a single appendix:
%\appendix[Proof of the Zonklar Equations]
% or
%\appendix  % for no appendix heading
% do not use \section anymore after \appendix, only \section*
% is possibly needed

% use appendices with more than one appendix
% then use \section to start each appendix
% you must declare a \section before using any
% \subsection or using \label (\appendices by itself
% starts a section numbered zero.)
%


\section*{Acknowledgement}


\IEEEPARstart{T}{HIS} report uses IEEE latex template \texttt{bare\_jrnl.tex}, which is a template designed for computer science related journal papers.


% Can use something like this to put references on a page
% by themselves when using endfloat and the captionsoff option.
\ifCLASSOPTIONcaptionsoff
  \newpage
\fi



% trigger a \newpage just before the given reference
% number - used to balance the columns on the last page
% adjust value as needed - may need to be readjusted if
% the document is modified later
%\IEEEtriggeratref{8}
% The "triggered" command can be changed if desired:
%\IEEEtriggercmd{\enlargethispage{-5in}}

% references section

% can use a bibliography generated by BibTeX as a .bbl file
% BibTeX documentation can be easily obtained at:
% http://www.ctan.org/tex-archive/biblio/bibtex/contrib/doc/
% The IEEEtran BibTeX style support page is at:
% http://www.michaelshell.org/tex/ieeetran/bibtex/
%\bibliographystyle{IEEEtran}
% argument is your BibTeX string definitions and bibliography database(s)
%\bibliography{IEEEabrv,../bib/paper}
%
% <OR> manually copy in the resultant .bbl file
% set second argument of \begin to the number of references
% (used to reserve space for the reference number labels box)
\begin{thebibliography}{1}

\bibitem{dragon}
Alfred V. Aho, Monica S. Lam, Ravi Sethi and Jeffrey D. Ullman, \emph{Compilers Principles Techniques and Tools}, 2nd edition. 

\bibitem{tiger}
Andrew W. Appel, \emph{Modern Compiler Implementation in Java}, 2nd edition. 

\bibitem{alss}
James R. Larus, \emph{Assemblers, Linkers, and the SPIM Simulator}.

\bibitem{ls}
Massimiliano Poletto and Vivek Sarkar, \emph{Linear Scan Register Allocation}.

\end{thebibliography}

% biography section
% 
% If you have an EPS/PDF photo (graphicx package needed) extra braces are
% needed around the contents of the optional argument to biography to prevent
% the LaTeX parser from getting confused when it sees the complicated
% \includegraphics command within an optional argument. (You could create
% your own custom macro containing the \includegraphics command to make things
% simpler here.)
%\begin{biography}[{\includegraphics[width=1in,height=1.25in,clip,keepaspectratio]{mshell}}]{Michael Shell}
% or if you just want to reserve a space for a photo:


% You can push biographies down or up by placing
% a \vfill before or after them. The appropriate
% use of \vfill depends on what kind of text is
% on the last page and whether or not the columns
% are being equalized.

%\vfill

% Can be used to pull up biographies so that the bottom of the last one
% is flush with the other column.
%\enlargethispage{-5in}



% that's all folks
\end{document}


